\documentclass[10pt, usenames, dvipsnames, table]{beamer}
\usetheme{Berkeley}
\usecolortheme{seagull}
\usepackage{graphicx}
\graphicspath{ {../images/} }
\usepackage{array}
\RequirePackage{fix-cm}
\usepackage{colortbl}
\usepackage{hyperref}
\usepackage{graphbox}
\usepackage{algorithm}
\usepackage{algpseudocode}
\usepackage{minted}
\setminted{} % numbers=right, bgcolor=lightgray}


\title{The Scala Programming Language}
\author{Troy Hut and Benjamin Killeen}
\date{}

\begin{document}

\begin{frame}
  \titlepage{}
\end{frame}

\section{Introduction}
\begin{frame}
  \frametitle{Introduction}
  Scala is:
  \begin{itemize}
  \item<2-> Object oriented
  \item<3-> Functional
  \item<4-> Frustratingly well-typed
  \end{itemize}
\end{frame}

\begin{frame}
  \frametitle{Scala is \textbf{Scalable}}
  \begin{figure}
    \centering
    \includegraphics[width=0.8\linewidth]{scala_classes}
    \caption{The Scala class hierarchy.}
    \label{fig:class_hierarchy}
  \end{figure}
\end{frame}

\section{Object Oriented}

\subsection{Natural Numbers}
\begin{frame}
  \frametitle{Traits, Objects, and Classes}
  \inputminted{Scala}{../examples/ExampleNat.scala}
  \pause{}
  \inputminted{Scala}{../examples/ExampleZero.scala}
  \pause{}
  \inputminted{Scala}{../examples/ExampleSucc.scala}  
\end{frame}

\section{Functional}
\begin{frame}
  \frametitle{The \mintinline{Scala}|nat| Package}
  \inputminted[firstline=1, lastline=30, fontsize=\tiny]
  {Scala}{../examples/Nat.scala}
\end{frame}



\begin{frame}
  \frametitle{Variance}
  % todo: describe variance a bit, use the example
\end{frame}

\subsection{Ordered List}
\begin{frame}
  \frametitle{Ordered List}
  \inputminted[firstline=1, fontsize=\tiny]
  {Scala}{../examples/OrderedList.scala}
\end{frame}

\end{document}
%%% Local Variables:
%%% mode: latex
%%% TeX-master: t
%%% TeX-command-extra-options: "-shell-escape"
%%% End:
